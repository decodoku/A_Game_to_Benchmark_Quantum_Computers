\documentclass[aps,prl,twocolumn,showpacs,preprintnumbers]{revtex4-1}

% Package(s) to include
\usepackage{psfrag,graphicx}
\usepackage{dcolumn}
\usepackage{amsmath,amssymb}
\usepackage{bm}
\usepackage{amsfonts,amssymb,amsmath}        % for math symbols.
\usepackage{epstopdf}
\usepackage{mathtools}
\DeclarePairedDelimiter{\ceil}{\lceil}{\rceil}

\newcommand{\be}{\begin{equation}}
\newcommand{\ee}{\end{equation}}
\newcommand{\bq}{\begin{eqnarray}}
\newcommand{\eq}{\end{eqnarray}}

\newcommand{\twoDEG}{\textsc{2deg}}
\newcommand{\SOI}{\textsc{soi}}

\newcommand{\ket}[1]{\left | \, #1 \right\rangle}
\newcommand{\bra}[1]{\left \langle #1 \, \right |}

\bibliographystyle{apsrev}

\begin{document}
\title{Benchmarking of quantum processors with random circuits}
\author{James R. Wootton}
\affiliation{Department of Physics, University of Basel, Klingelbergstrasse 82, CH-4056 Basel, Switzerland}


\begin{abstract}

Quantum processors with sizes in the 10-100 qubit range are now commonplace. However, with increased size comes increased complexity for benchmarking. The effectiveness of a given device may vary greatly between different tasks, and won’t always be  easy to predict from single and two qubit gate fidelities. For this reason, it is important to assess processor quality for a range of important tasks. In this work we propose and implement tests based on random quantum circuits. These are used to evaluate four different superconducting qubit devices, with sizes from 5 to 19 qubits, from three hardware manufacturers: IBM Research, Rigetti and Alibaba.


\end{abstract}


\pacs{}

\maketitle


\section{Introduction}



\section{Generation of random circuits}



\section{Figures of merit}

\subsection{Error mitigation}



\section{Results}


\subsection{5 and 16 qubit IBM devices}


\subsection{19 qubit Rigetti device}


\subsection{11 qubit Alibaba device}



\section{Conclusions}


%\begin{figure}[t]
%\begin{center}
%{\includegraphics[width=\columnwidth]{device.png}}
%\caption{\label{device} The layout of the \emph{ibmqx3} device, with the numbering of qubits used in this study. Lines connect pairs of qubits for which a CNOT can be performed. Thick black lines are show for the CNOTs used in our implementation of the repetition code.
%}
%\end{center}
%\end{figure}




\section{Acknowledgements}

This work was supported by the Swiss National Science Foundation and the NCCR QSIT.

The IBM Quantum Experience was used to produce results for this work. The views expressed are those of the authors and do not reflect the official policy or position of IBM or the IBM Quantum Experience team.


\bibliography{refs}


\end{document}
